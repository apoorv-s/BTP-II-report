\chapter{Introduction}
The project is aimed towards the study of novel structure and materials at macro as well as micro level, understanding their structure and behavior, and ways in which they can altered to achieve required functionalities. The project started with the study of deployable structures and eventually diverged to touch several exciting fields which are described below.

\section{Deployable Structure}
Deployable structures are defined as the structures capable of undergoing large changes in shape. These structures are also often referred to as foldable, reconfigurable, auxetic, extendible or expandable structures. Deployable structures are used extensively in the Aerospace industry where large structures such as satellites, antennas, solar arrays etc need to be packaged in compact volumes and deployed when in outer space. There are two manners in which development of deployable structure is undertaken. The first one involves the use of structural components, the components may be rigid or flexible or combination of both. The second approach employs the concepts of origami, biomimetics and other form inspiring sources and is termed this approach is termed as Generative technique.\cite{rivas2015deployable}


According to Maxwell's Lemma, the lightest (and therefore most efficient) structure separates compressive and tensile elements, this leads to a special class of Deployable structures called Tensegrity, Tensional integrity or Floating compression. Such structures consist of compression members floating in a net of tension members and the compression members do not usually touch each other. Since there is no bending, such structures are extremely efficient and have high rigidity-to-weight ratio. \cite{pellegr}

\section{Mechanisms and States of self stress}
Pin jointed frameworks can be classified into truss and mechanisms based on their connectivity. While a truss is defined as a rigid structure capable of carrying loads when simply supported, according to classical definition, Mechanism is defined as a system of rigid elements connected through pins and joints capable of movement. These mechanisms are classified into several groups such as planar mechanisms, spatial mechanisms, spherical mechanisms and so on. This classification can also be examined in terms of Static and Kinematic Indeterminacy.\cite{Pelle}


The assemblies in which all member forces can be determined using the equations of equilibrium are classified as statically determinate structure. Another equivalent definition of static determinacy is that for a given system, the number of equations is equal to the number of unknowns and the coefficient matrix is non-singular. Similarly Kinematic determinate structures are the assemblies in which the position of the nodes can be determined exactly (on one side of the base plane) based on the lengths of bars. kinematically indeterminate structures are the assemblies in which position of nodes can not be determined uniquely and they have one or more mode of inextensional deformation. It means that such an assembly can distort without change in member lengths which essentially makes it a mechanism. In a similar manner, state of self stress is related to statically indeterminate structures. When an assembly has states of self stress, it's member can have forces without application of any external load. 


\section{Origami}
Origami is a Japanese word derived from ori meaning "folding", and kami meaning "paper" and stands for the art of paper folding. Another term closely associated with Origami is Kirigami, unlike origami cutting of the paper is allowed in Kirigami. The Origami technique comprises of several basic basic folds like valley and mountain folds, pleats, reverse folds, squash folds, and sinks, which are used to generated complex patterns and shapes using a single sheet of paper.


The art of Origami has found several applications in the industries such as medical and space industry and continues to inspire new techniques in other fields. The techniques of origami are currently being investigated for development of new materials and tools such as Metamaterials, Sandwich panels, drones and robots.

\section{Mechanical Metamaterials}
Metamaterials are defined as the materials which are engineered for the properties that are not present in naturally occurring materials. Metamterials gain their property from their internal structure and connectivity rather than the materials properties. Based on the property with which a metamaterial is concerned, it is classified into one of the several sub-divisions such as Electromagnetic Metamaterials, Thermo-Electric Metamaterials and so on. Mechanical metamaterials are concerned with mechanical properties of matter and focuses on achieving unusual values for mechanical parameters such as density, Poisson's ratio, and compressibility. Recent development in the field are associated with the development of exotic functionalities such as pattern and shape transformation in response to mechanical forces, unidirectional guiding of motion and waves and reprogrammable stiffness\cite{Berto, Surj}.

\section{Form Finding}
The primary motive of the form-finding process is to identify the geometry that sustains the load coming on the structure most efficiently or the shape a structure takes under a given load. In cases when elasticity and geometric constraints are intertwined, for example structure such as elastic gridshells which buckle by design, actuated shapes are difficult to predict using classical methods. Such cases suggest the presence of multi-stable states and the analysis requires inclusion of higher modes of the structure. Such techniques can applied in the opposite direction as well to determine the boundary conditions and forces that need to be applied on the structure to achieve the desired shape.\cite{Baek75} 

\section{Elastic Bilayers}
Elastic Bilayers stands for shape shifting thin sheets made up of active materials that respond to stimuli such as heat, light and humidity. Such layers are designed by solving the geometric inverse problem of determining the growth factors and directions for a given isotropic elastic bilayer to grow into a target shape by posing and solving an elastic energy minimization problem. Such techniques aid in engineering complex functional shapes in tissues, and actuation systems in soft robotics.\cite{va}

\section{Topological Mechanics}
In Mathematics, Topology is defined as the study of properties of a geometric object that are preserved under continuous deformations, such as stretching, twisting, crumpling and bending. The concepts from Topology are aiding the development of novel materials such as topological insulators, and topological photonics. In the recent times, the concepts from electronic topological states are being applied to mechanics to identify topological mechanical properties. Topological mechanics encompasses the study of topological phonon modes of the material, which has applications in development of mechanical insulators and metamaterials with topologically protected mechanical properties.\cite{Ma, Rock, Baardink489, Che}

\section{Lattices \& Non-Affine Deformations}
Lattice is an ordered arrangement of particles which repeats infinitely in all dimensions of the lattice. The unit cell of a lattice is defined as the smallest repeating unit having the full symmetry of the lattice and Structure of a lattice is described by its unit cell. A lattice may have more than one unit cell which when translated can generate entire lattice. A special class of lattices are termed as Maxwell lattices. Maxwell lattices are mechanical frames having average coordination number equal to twice their spatial dimension, this leaves them on verge of mechanical instability. Fourier Transform of these lattices also results in lattices, which are called Reciprocal Lattices and present the lattice in reciprocal space. Reciprocal Lattices are used for determining the phonon modes which helps in design of materials with topologically protected material properties.


Changes in the metric properties of a continuous body is defined as deformation, this indicates that a curve drawn on original body will change its length after the body is deformed. Usually deformations are affine which means that the deformations can be described in terms of affine transformation, or equivalently local strain in a sample after deformation is identical everywhere and equal to the macroscopic strain. All deformations other than affine deformations are termed as non-affine deformation. One of the principal sources of non-affinity is a space or time dependent elastic constant. The local environment in a disordered solid varies in space, depending crucially on local connectivity or coordination such that the local displacement $\textbf{u}$ may not be simply related to the applied stress $\boldsymbol{\sigma}$. Such non-affine displacements are present even at zero temperature, are material dependent, and vanish only for homogeneous crystalline media without defects\cite{Gang}.

